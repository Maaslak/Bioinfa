\documentclass{article}
\title{Sprawozdanie 2 \\ Testowanie opracowanej metody heurystycznej}
\date{2018-06-01}
\author{Mateusz Babiaczyk, Bartosz Nawrotek}
\usepackage[utf8]{inputenc}
\usepackage[T1]{fontenc}
\usepackage{geometry}
 \geometry{
 a4paper,
 total={170mm,257mm},
 left=20mm,
 top=20mm,
 }

\begin{document}
\maketitle
\section{Zmiany w algorytmie}  
Po zaimplementowaniu algorytmu i zauważeniu jego słabych osiągnięć, doszliśmy do wniosku by wprowadzić zmiany w naszym algorytmie. 
\subsection{Mutacje}
\subsection{Krzyżowanie}
Krzyżowanie zaczyna się w dokładnie taki sam sposób jak było w naszym pierwotnym algorytmie, a mianowicie od pewnego wylosowanego przedziału przepisuje oligonukleotydy do nowo tworzonego osobnika (kopiuje wycinek i wkleja go do nowego osobnika) z wybranego osobnika z populacji rodzicielskiej. Następnie uzupełniany jest koniec osobnika wartościami z innego osobnika z populacji rodzicielskiej, uważając oczywiście by dany oligonukleotyd nie został powtórzony. W ten sam sposób zostaje uzupełniony początek osobnika z nowej populacji. \\ Wszystkie oligonukleotydy które nie zostały dodane (na wskutek dodawania ich z innego osobnika który za punktem cięcia mógł mieć oligounkloeotyd taki sam jak drugi osobnik użyty do krzyżowania pomiędzy punktami cięcia) zostają dodane, każdy osobno, w miejsce w którym funkcja celu będzie najniższa. Dzięki takiemu nakierowaniu, nadal mieliśmy pewną dużą losowość przez krzyżowanie między losowymi punktami cięcia (która jest ważna w algorytmie genetycznym), a jednocześnie algorytm szybciej zbiegał do wartości optymalnych, tym samy dając lepsze rezultaty w krótszym czasie.
\subsection{Kodowanie}
\section{Testy}
\section{Wnioski}
\end{document}


