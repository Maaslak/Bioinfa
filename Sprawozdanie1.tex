\documentclass{article}
\title{Sprawozdanie 1 \\ Teoretyczne opracowanie metody heurystycznej}
\date{2018-04-08}
\author{Mateusz Babiaczyk, Bartosz Nawrotek}
\usepackage[utf8]{inputenc}
\usepackage[T1]{fontenc}
\usepackage{geometry}
 \geometry{
 a4paper,
 total={170mm,257mm},
 left=20mm,
 top=20mm,
 }

\begin{document}
\maketitle
\section{Wprowadzenie teoretyczne}
Do rozwiązania problemu zastosowaliśmy algorytm genetyczny.
\subsection{Kodowanie}
Osobniki są kodowane jako wektory liczb całkowitych w których indeks oznacza kolejność występowania oligonukleotydu w sekwencji:
\begin{equation}
	X = [ x_{1}, x_{2}, \ldots , x_{m}] , 
\end{equation}
gdzie:
\begin{enumerate}
	\item $m$ - wielkość zbioru dostępnych oligonukleotydów
	\item $x_{i}$ dla $i \in \langle1, m\rangle$ - indeks oligonukleotydu w liście dostępnych oligonukleotydów
\end{enumerate}
Jako rozwiązanie będziemy traktować sekwencje zbudowaną z oligonukleotydów począwszy od $x_{1}$ nie przekraczającą długości $n$.
\subsection{Funkcja oceny}
Jako minimalizowaną funkcję oceny osobnika przyjęliśmy następującą postać addytywną:
\begin{equation}
	f_{min}(X) = \sum_{i = 1}^{i = k - 1}{[2len(x_{i}, x_{i+1}) - l + 1]} + \sum_{i = k}^{i = m - 1}{[len(x_{i}, x_{i+1})- l + 1]}
\end{equation}
gdzie:
\begin{enumerate}
	\item $k$ - taka liczba całkowita, dla której długość sekwencji $[x_{1}, x_{2}, \ldots , x_{k}]$ będącej złożeniem $k$ pierwszych oligonukleotydów osobnika $X$ jest mniejsza lub równa $n$
	\item $len(x, y)$ jest długością sekwencji otrzymanej z połączenia oligonukleotydów $x$ oraz $y$
	\item l - długość oligonukleotydu
\end{enumerate}
Przyjmując funkcję kosztu powyższej postaci pragnęliśmy zwrócić szczególną uwagę na pierwsze k oligonukleotydów z osobnika, które zawierają szukane rozwiązanie. Druga część sumy ma na celu nie losowe uporządkowanie pozostałych oligonukleotydów, które będą wykożystane w operacji krzyżowania
\end{document}
